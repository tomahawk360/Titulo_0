\secnumbersection{MARCO CONCEPTUAL}

\subsection{TELESCOPIO VLT}
El VLT, sigla para “Very Large Telescope”, es un telescopio ubicado en el Cerro Paranal a 2635 metros de altura, como parte de la “European Southern Observatory” (o conocida por su sigla ESO). 
Los fines bajo los cuáles se construyó el VLT corresponden a los siguientes \cite{eso1998vlt}:
\begin{itemize}

    \item El mayor área de colecta posible, según los recursos disponibles.
    \item La mayor cobertura de longitud de onda, con tal de explotar completamente todas las ventanas atmosféricas.
    \item Máxima flexibilidad y amplia diversificación instrumental, permitiendo múltiples usos de las instalaciones, incluyendo observaciones simultáneas de múltiples longitudes de onda.
    \item Capacidad limitada por la difracción de la mayor línea de base posible.
    \item Optimización de los procedimientos de operaciones científicas, con tal de permitir la explotación, completa y en tiempo real, de la calidad astronómica del sitio y garantizar un máximo retorno científico.

\end{itemize}

El VLT se compone de una red de 4 telescopios principales idénticos, denominados como “Unitary Telescopes” (o por sus siglas UT), 4 telescopios auxiliares, denominados como “Auxiliary Telescopes” (o por sus siglas AT), el  Interferómetro VLT (por sus siglas en inglés VLTI), 1 VISTA y 1 VST.

Cada UT posee un espejo principal (denominado M1) de forma cóncava con 8,2 metros de diámetro, instalado en un montura de 22 metros de largo, 10 metros de ancho y 20 metros de alto. Dicha montura permite al espejo moverse según un eje de coordenadas horizontal, esto es, según azimut (eje horizontal) y altitud (eje vertical) \cite{eso1998vlt}.

El M1 está compuesto de espejos más pequeñas, distribuidas en forma de dona. Bajo M1, repartidos en 6 anillos concéntricos, se encuentran 150 actuadores de fuerza axiales; estos son, propulsores hidráulicos y/o neumáticos, donde cada actuador se encuentra debajo de una espejo pequeña respectiva. Estos actuadores dan a M1 una determinada forma óptica, determinada por el patrón de fuerza presentado por los actuadores \cite{eso1998vlt}.

El M2 consiste de una espejo de 0.9 metros de diámetro, montada a una distancia de aproximadamente 12.3 metros de M1 a lo largo del eje azimutal, con la espejo de M2 apuntando hacia M1. El M2 además está montado sobre un mecanismo electromecánico que sujeta y controla su inclinación \cite{eso2011m2}.

El M3 consiste de una espejo elíptica de 1.24 metros de diámetro mayor con 0.86 metros de diámetro menor. Este se ubica dentro de una torre, posicionada en el orificio central del M1. Este puede rotar alrededor de su eje azimutal \cite{eso2011m1}.

Los ATs son telescopios con espejos de 1,8 metros de diámetro, encargados de complementar la luz obtenida por los UTs cuando operan bajo el VLTI.

El VLTI corresponde al arreglo de los UTs y ATs funcionando de forma coordinada, recombinando la luz obtenida por cuatro telescopios (los cuales pueden ser UTs y/o ATs) de forma simultánea, cada uno encargado de recombinar secciones específicas del espectro electromagnético. Con este sistema, el diámetro total disponible para captar luz es igual a distancia entre los telescopios \cite{eso1998vlt}.

\subsection{SISTEMA DE ÓPTICA ACTIVA}
La Óptica Activa es un sistema integrado en el M1, encargado de corregir las aberraciones y degradación en la calidad de imágen generadas por las ópticas del espejo \cite{eso1998vlt}. 

Estas aberraciones suelen ser causadas por la sensibilidad de este a perturbaciones ambientales, cómo las distorsiones térmicas, turbulencia atmosférica, deformación de espejo por ráfagas de viento, errores de manufactura y mantenimiento del telescopio, entre otros. Esta sensibilidad a su vez es causada por la baja proporción entre el grosor y el diámetro del espejo \cite{wilson1987active}.

El ciclo básico del sistema de Óptica Activa se ilustra en la imágen del Anexo 1.

El sensor de frente de onda Shack-Hartmann toma una estrella desde el cielo y la usa como referencia offset, con la cuál monitorea constantemente la calidad óptica de la imágen mediante análisis. Cuando se calcula una desviación considerable de la imágen de su calidad óptima, se descompone dicha desviación en contribuciones ópticas simples (grados de astigmatismo, desenfoque, entre otros.) y posteriormente calcula las correcciones al patrón de fuerza presente en los actuadores de fuerza de M1 \cite{eso1998vlt}.
Este nuevo patrón de fuerza cambia la forma de M1, con la cuál se obtiene una nueva calidad de la imágen captada. Luego, con esta imágen, el sensor de frente de onda Shack-Hartmann vuelve a calcular la desviación a la calidad óptima, y si la desviación calculada es considerable, se repite el proceso hasta que se logre la calidad óptima \cite{eso1998vlt}\cite{wilson1987active}.

\subsection{SENSOR DE FRENTE DE ONDA}
Un sensor de frente de onda es un dispositivo electrónico diseñado para capturar los frentes de onda de una fuente, ya sea en el espectro visible o en infrarrojo. El VLT usa CCD Shack-Hartmann como sensores de frente de onda, con cada UT poseyendo uno en un brazo mecánico específico para los sensores.\cite{eso1998vlt}

El objetivo del uso de sensores Shack-Hartmann es la medición de distorsiones del frente de onda capturado desde la fuente, sobre las cuáles se realiza el proceso de Óptica Activa.\cite{eso1998vlt}

\subsection{LOGS}
Logging se refiere al proceso de registrar diferentes eventos y actividades que ocurren dentro de un sistema de software \cite{jayathilake2011mind}. Estos registros suelen almacenarse en archivos para su posterior análisis, tanto por parte de desarrolladores como de operadores externos.

Los registros de log son los único tipo de datos que registran, valga la redundancia, información de la operación interna de un sistema de software, por lo que su rol en la industria es importante \cite{ma2023automatic}.

\subsection{ANÁLISIS DE LOGS}
Originalmente, el análisis de registros de logs era realizado por desarrolladores con el fin de trazar el flujo de ejecución del sistema de software, identificar excepciones y potenciales errores. \cite{jayathilake2011mind}

Actualmente este enfoque se ha expandido a casos de uso en otros servicios en la industria \cite{ma2023automatic}, debido a que, por la naturaleza de la información contenida en los registros de log, su análisis permite a los operadores detectar, diagnosticar e incluso predecir errores que puedan afectar la disponibilidad y el rendimiento del sistema de software \cite{jayathilake2011mind}.

Ejemplos de estos nuevos casos de uso incluye suplementar información para la detección de intrusiones en redes, recopilación de logs de gran tamaño en forensia digital, entre otros. \cite{ma2023automatic}

En el pasado, durante la prevalencia del enfoque original, el análisis de registros de log era realizado aplicando revisiones visuales y reglas construidas manualmente. Sin embargo, la complejidad de los sistemas de software actuales ha llevado a la complejización de sus respectivos registros de logs, por lo que ya no es posible depender solamente de los métodos anteriormente mencionados. \cite{ma2023automatic}

Por esto, en los últimos años se ha desarrollado ampliamente el área del análisis automatizado de registros de logs, mejorando su eficiencia y exactitud mediante la aplicación de tecnologías distribuidas y técnicas de machine learning. \cite{ma2023automatic}

\subsection{ESTRUCTURACIÓN DE LOGS}
Los registros de logs generalmente poseen una composición demasiado compleja como para ser interpretada de forma directa y manual; sin el acceso de conocimiento profesional, es difícil seleccionar de forma manual las reglas apropiadas para la comprensión de los registros de log \cite{ma2023automatic}. Por esta condición es que se refiere a que los registros de log sean “No Estructurados” o “Semi Estructurados”. 

Además, el gran tamaño de los archivos de log promedio también se vuelve un problema para el análisis manual de los registros de log \cite{ma2023automatic}.

Debido a esto, durante los últimos años, se han desarrollado herramientas, procedimientos y frameworks para el análisis automático de registros de log, y una parte considerable de los esfuerzos realizados se enfocan en la “Estructuración” de los registros de log.

Actualmente, el workflow de análisis automático de registros de log se divide en 2 etapas centrales \cite{ma2023automatic}:

\begin{itemize}
    \item Log Parsing: Se toman los registros semi-estructurados de log y se generan plantillas a partir de estos. Una plantilla es una sentencia estructurada que se repite entre varios registros de log, dividiéndose en tokens estáticos y valores dinámicos. \cite{ma2023automatic}

    \item Feature Extraction: Se aplican las plantillas generadas sobre los registros de log para obtener las características, esto es, las variables dinámicas, de los mismos. \cite{ma2023automatic}

\end{itemize}