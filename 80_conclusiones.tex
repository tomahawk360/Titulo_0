\secnumbersection{CONCLUSIONES}

En esta memoria se logró abordar la problemática planteada, con el desarrollo de un procedimiento que extrae información de los registros de log de los telescopios UTs, en relación con el sistema de Óptica Activa, y los disponga en datasets de datos.

Para lograr esto, fue necesario estudiar varios métodos de extracción de datos y estructuración de logs, intentando implementar métodos de la forma más fiel a la fuente original, hasta que se decantó por tomar las bases de estos métodos para desarrollar uno propio.

Igualmente, durante una parte considerable del desarrollo, se barajo el uso de dos métodos de extracción: una usando Expresiones Regulares, y otra usando una librería especializada de Python. Si bien ambas opciones son válidas, para el caso de esta memoria se demostró que la opción de Expresiones Regulares era la más apta.

El algoritmo actual genera datasets con una cantidad considerable de datos, extraídos con la seguridad total de que se generan durante las observaciones del telescopio. Sin embargo, el algoritmo deja una alta cantidad de líneas de log sin procesar, y no se puede corroborar que todas las líneas de log descartadas no posean datos de distribución de fuerzas o de condiciones ambientales del telescopio. Por lo mismo, un punto crucial para estudio futuro es la búsqueda de nuevos métodos, o mejorar los ya existentes, con tal de ampliar el rango de captura de líneas de log para su análsis y extracción de datos.

Otro enfoque de posible interés para estudio futuro es la generalización de este algoritmo, ya que actualmente está enfocado al sistema de Óptica Activa. Se debe evaluar si la misma arquitectura puede servir para la extracción de datos orientados a otras áreas de la operación de los UTs, como por ejemplo la detección de errores en tiempo real.