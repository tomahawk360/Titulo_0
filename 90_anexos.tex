\secnumberlesssection{ANEXOS}

\begin{itemize}
    \item Anexo 1:
        \\
        \vspace{3.5cm}
        \includegraphics[width=6cm,height=9cm]{figures/actopt2.jpg} \\
        \vspace{3.5cm}


    \item Anexo 2:
        \\
        \vspace{3.5cm}
        \includegraphics[width=12cm,height=18cm]{figures/log_vertical.png} \\
        \vspace{3.5cm}

    \item Anexo 3:
        \\
        \\
        \vspace{3.5cm}
        \includegraphics[width=12cm,height=16cm]{figures/flow_diagram.png} \\
        \vspace{3.5cm}

    \item Anexo 4:
        \\
        \\
        \vspace{3.5cm}
        \includegraphics[width=2cm,height=16cm]{figures/log_f_dist.png} \\
        \vspace{3.5cm}
\end{itemize}


\begin{itemize}
    \item Diccionario 1: Frente de Onda

    El frente de onda es una superficie imaginaria que representa los puntos correspondientes de una onda que vibra al unisono. Cuando ondas idénticas con un origen en común viajan a través de un medio homogéneo, los senos y picos correspondientes a cada uno se mantienen en fase \cite{britannica2022front}. 

    \item Diccionario 2: Sensor de frente de onda

    Es un sensor diseñado para calcular las aberraciones de una imágen mientras se toma la misma. Generalmente las aberraciones son producidas por ladeos en los frentes de onda, presentes en imágenes que capturan objetos al otro lado de la atmósfera \cite{platt2001sh}
\end{itemize}