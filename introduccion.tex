\secnumberlesssection{INTRODUCCIÓN}

Debe proporcionar a un lector los antecedentes suficientes para poder contextualizar en general la situación tratada, a través de una descripción breve del área de trabajo y del tema particular abordado, siendo bueno especificar la naturaleza y alcance del problema; así como describir el tipo de propuesta de solución que se realiza, esbozar la metodología a ser empleada e introducir a la estructura del documento mismo de la memoria.

En el fondo, que el lector al leer la Introducción pueda tener una síntesis de cómo fue desarrollada la memoria, a diferencia del Resumen dónde se explicita más qué se hizo, no cómo se hizo.
