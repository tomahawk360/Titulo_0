\secnumbersection{MARCO CONCEPTUAL}

Se debe describir la base conceptual o fundamentos en los que se basa tu memoria, es decir, todos los conceptos técnicos, metodologías, herramientas, etc. que están involucradas en la solución propuesta. En el fondo esta parte permite precisar y delimitar el problema, estableciendo definiciones para unificar conceptos y lenguaje y fijar relaciones con otros trabajos o soluciones encontradas por otros al mismo problema evitando así plagios o repetir errores ya conocidos o abordados por otros.

En esta parte es importante relacionar estos conceptos con la memoria y es fundamental utilizar referencias bibliográficas (o de la web) recientes, por ejemplo \cite{gettelfinger2004will}.
